\documentclass[a4paper,notitlepage,11pt]{article}

\usepackage{pscyr}
\usepackage[T2A]{fontenc}
\usepackage[utf8]{inputenc}
\usepackage[english,russian]{babel}
\usepackage[russian]{olymp}
\usepackage{tikz}
\usepackage{amssymb}
\usepackage{amsmath}
\usepackage{verbatim}
\usepackage{enumerate}
\usepackage{systeme}
\usepackage[makeroom]{cancel}
\usepackage{epsfig}
\usepackage{epigraph}
\usepackage{tikz}
\usepackage{float}
\usepackage{wrapfig} % for having text alongside pictures
\usepackage{listings}
\usetikzlibrary{arrows}
\usepackage{color}

\definecolor{dkgreen}{rgb}{0,0.6,0}
\definecolor{gray}{rgb}{0.5,0.5,0.5}
\definecolor{mauve}{rgb}{0.58,0,0.82}

\lstset{frame=tb,
  language=Python,
%  aboveskip=3mm,
%  belowskip=3mm,
  showstringspaces=false,
  columns=flexible,
  basicstyle={\small\ttfamily},
  numbers=none,
  numberstyle=\tiny\color{gray},
  keywordstyle=\color{blue},
  commentstyle=\color{dkgreen},
  stringstyle=\color{mauve},
  breaklines=true,
%  breakatwhitespace=true,
  tabsize=2, 
  inputencoding=utf8,
  extendedchars=true
}

% IF YOU DON'T HAVE THIS FONT PACKAGE, JUST COMMENT THE NEXT THREE LINES!
\usepackage{pscyr}
\renewcommand{\rmdefault}{fha}
\renewcommand{\sfdefault}{fha}
% OR GET IT HERE: http://www.tex.uniyar.ac.ru/package/fonts/pscyr/

\renewcommand{\t}[1]{\ifmmode{\mathtt{#1}}\else{\texttt{#1}}\fi}
\newcommand{\s}[1]{`\t{#1}'}
\newcommand{\q}[1]{<<\t{#1}>>}
\newcommand{\eps}{\varepsilon}
\renewcommand{\t}{\texttt}
\renewcommand{\le}{\leqslant}
\renewcommand{\leq}{\leqslant}
\renewcommand{\ge}{\geqslant}
\renewcommand{\geq}{\geqslant}
\newcommand{\pow}{\^\relax}
\newcommand{\bs}{\mbox{$\backslash$}}
\newcommand{\seq}[2]{${#1}_{1}$, ${#1}_{2}$, $\ldots$, ${#1}_{#2}$}
\newcommand{\seqop}[3]{${#1}_{1} #3 {#1}_{2} #3 \ldots #3 {#1}_{#2}$}
\def\bbZ{\mathbb{Z}}
\DeclareMathOperator{\perm}{perm}

\contest
{Сессия по информатике, центр <<Интеллект>>}
{Разбор задач по геометрии}
{пятница, 10 июля 2015 года}

\binoppenalty=10000
\relpenalty=10000
%\parindent=0cm

\newcommand{\seriesshortname}{G}
\newcommand{\letterenumi}{{\alph{enumi}}}
\newcommand{\partlabelenumi}{{\Huge \texttt{\seriesshortname\letterenumi}}}
\renewcommand{\labelenumi}{{\Huge \texttt{\partlabelenumi.}}}
\newcommand{\infile}{\q{\seriesshortname\letterenumi.in} }
\newcommand{\outfile}{{\q{\seriesshortname\letterenumi.out}} }
\newcommand{\backref}[1]%
{\addtocounter{enumi}{-#1}\partlabelenumi\addtocounter{enumi}{+#1}}

\begin{document}

\centerline{\Huge Пересечение двух окружностей}

\begin{tikzpicture}[scale=3,cap=round]
  % Local definitions
  \def\costhirty{0.8660256}

  % Colors
  \colorlet{anglecolor}{green!50!black}
  \colorlet{color1}{red}
  \colorlet{color2}{blue}
  \colorlet{sincolor}{red}
  \colorlet{tancolor}{orange!80!black}
  \colorlet{coscolor}{blue}

  % Styles
  \tikzstyle{axes}=[]
  \tikzstyle{important line}=[very thick]
  \tikzstyle{information text}=[rounded corners,fill=red!10,inner sep=1ex]

  % The graphic
  \draw[style=help lines,step=0.5cm] (-1.4,-1.4) grid (1.4,1.4);
  
  \draw[style=important line,color1] (0,0) circle (1.2cm);
 
  \draw[style=important line,color2] (0.8, 0.4) circle (0.7cm);

  \begin{scope}[style=axes]
    \draw[->] (-1.5,0) -- (1.5,0) node[right] {$x$};
    \draw[->] (0,-1.5) -- (0,1.5) node[above] {$y$};

    \foreach \x/\xtext in {-1, -.5/-\frac{1}{2}, 1}
      \draw[xshift=\x cm] (0pt,1pt) -- (0pt,-1pt) node[below,fill=white]
            {$\xtext$};

    \foreach \y/\ytext in {-1, -.5/-\frac{1}{2}, .5/\frac{1}{2}, 1}
      \draw[yshift=\y cm] (1pt,0pt) -- (-1pt,0pt) node[left,fill=white]
            {$\ytext$};
  \end{scope}

  \draw (0.8,0) -- (intersection of 0,0--30:1cm and 0.8,0--0.8,1) coordinate (t);
    
  \draw (0,0) -- node [above left]
    {
      $\displaystyle \sqrt{x_2^2 + y_2^2}$
    } (t);

  \draw[xshift=1.85cm] node [right,text width=6cm,style=information text]
    {
      \textbf{Геометрия}

1. Геометрические объекты (понимаем, что они там есть: точки, прямые, отрезки, окружности)

2. Рисунок (упрощаем: перемещаем в начало координат, выкидываем лишнее)

3. Уравнения + неравенства (вид уравнения - вид объекта, параметы - конкретный объект)

4. Решаем уравнения + проверяем неравенства (+ точность вычислений)

5. Частные случаи => Интерпретации.

    };
\end{tikzpicture}

\systeme{
x^2 + y^2 = {R_1^2},
(x-x_2)^2 + (y-y_2)^2 = {R_2^2}
}

Раскрываем скобки во втором уравнении:
$$x^2 - 2x{x_2} + x_2^2 + y^2 - 2y{y_2} + y_2^2 = {R_2^2}$$

Вычитаем из второго уравнение первое и получаем уравнение прямой $ax + bx + c = 0$:
$$\cancel{x^2} - 2x{x_2} + x_2^2 + \cancel{y^2} - 2y{y_2} + y_2^2 - \cancel{x^2} - \cancel{y^2} = {R_2^2} - {R_1^2}$$

$$\underbrace{-2{x_2}}_{= a}x + \underbrace{- 2{y_2}}_{= b}y + \underbrace{x_2^2 + y_2^2 + {R_1^2} - {R_2^2}}_{= c} = 0$$

Из $ax + by + c = 0$ выражаем $y = \frac{-c-ax}{b}$: 
$x^2 + (\frac{-c-ax}{b})^2 y^2 = R_1^2$

Раскрываем скобки и домножаем на $b^2$

$$\underline{\underline{{b^2}{x^2}}} + c^2 + \underline{2cax} + \underline{\underline{{a^2}{x^2}}} = {R_1^2}{b^2}$$

Квадратное уравнение относительно $x$:

$$\underbrace{(a^2+b^2)}_{A}x^2 + \underbrace{2ca}_{B}x + \underbrace{(c^2 - {R_1^2}{b^2})}_{C} = 0$$

Решаем квадратное уравнение:

Дискриминант: $D = B^2 - 4AC$


$$x_{1,2} = \frac{-B \pm \sqrt{D}}{2A}$$




\centerline{\Huge Треугольник и точка}

Заданы прямоугольные координаты $x_1, y_1$; $x_2, y_2$; $x_3, y_3$ вершин треугольника 
и координаты точки $x, y$. 
Определить, находится ли точка в треугольнике.

\begin{tikzpicture}[scale=1.5,cap=round]
  % Local definitions
  \def\costhirty{0.8660256}
 
  % Colors
  \colorlet{anglecolor}{green!50!black}
  \colorlet{color1}{red}
  \colorlet{color2}{blue}
  \colorlet{nodecolor}{black}
  \colorlet{sincolor}{red}
  \colorlet{tancolor}{orange!80!black}
  \colorlet{coscolor}{blue}
  
  % Styles
  \tikzstyle{axes}=[]
  \tikzstyle{important line}=[very thick]
  \tikzstyle{information text}=[rounded corners,fill=red!10,inner sep=1ex]

  % The graphic
  \draw[style=help lines,step=1cm] (-3.4,-3.4) grid (3.4,3.4);
  
  \begin{scope}[style=axes]
    \draw[->] (-3,0) -- (3,0) node[right,fill=white] {$x$};
    \draw[->] (0,-3) -- (0,3) node[above,fill=white] {$y$};

    \foreach \x/\xtext in {-3, -2, -1, 0, 1, 2, 3}
      \draw[xshift=\x cm] (0pt,1pt) -- (0pt,-1pt) node[below,fill=white]
            {$\xtext$};

    \foreach \y/\ytext in {-2, -1, 0, 1, 2}
      \draw[yshift=\y cm] (1pt,0pt) -- (-1pt,0pt) node[left,fill=white]
            {$\ytext$};
  \end{scope}
 
  \draw [style=important line,color2] 
     (-2,-1) coordinate (A) node[nodecolor,left,fill=white]{$A(x_1,y_1)$} -- 
     (1,3) coordinate (B) node[nodecolor,above,fill=white]{$B(x_2,y_2)$};
  \draw [style=important line,color2] (B) -- (2.5,-0.5) coordinate (C) node[nodecolor,right,fill=white]{$C(x_3,y_3)$};
  \draw [style=important line,color2] (C) -- (A);
  
  \draw (0.6,0.8) coordinate (P) node[nodecolor,right,fill=white]{$P(x,y)$} -- (A);
  \draw (P) -- (B);
  \draw (P) -- (C);
  
  
  \draw[xshift=3.6cm] node [right,text width=7cm,style=information text]
    {
Векторное произведение (ориентированная площадь):

$$v( \vec{a}, \vec{b} ) = \begin{vmatrix} a_x & a_y \\ b_x & b_y \end{vmatrix} = {a_x}{b_y} - {a_y}{b_x}$$

Разность векторов: 
$$\vec{AB} = (AB_x, AB_y) = (B_x - A_x, B_y - A_y)$$

Площадь треугольника $ABC$: $$S(A,B,C) = \left|\frac{v(AB,BC)}{2}\right|$$

    };
\end{tikzpicture}

Точка $P$ внутри треугольника $ABC$ если:
$$S(A,B,C) = S(A,B,P) + S(B,C,A) + S(C,A,P)$$

\centerline{\Huge Наибольшее расстояние}

 В некоторой стране $N$ городов и $M$ дорог. Каждая дорога представляет собой
 отрезок прямой, соединяющий два различных города, причём известно, что ни
 на одной дороге нет других городов, кроме тех, что задают её концы.
 Расстоянием между двумя городами считается минимальная суммарная длина дорог,
 по которым надо пройти, чтобы попасть из одного города в другой.
 Требуется вычислить наибольшее из таких расстояний.

Читаем исходные данные: координаты точек.

\begin{lstlisting}[language=Python]
n = int(input())
p = [] 
for i in range(n):
     x,y = map(int,input().split())
     p.append((x,y))
\end{lstlisting}

Читаем исходные данные - рёбра => матрица смежности $d_{i,j} = \sqrt{(x_i - x_j)^2 + (y_i - y_j)}$, нет ребра: $Inf = 10^{10}$
    
\begin{lstlisting}[language=Python]
Inf = 10**10
d = [[Inf]*n for i in range(n)]

m = int(input())        
for i in range(m):
    a,b = map(int,input().split())
    a -= 1
    b -= 1
    x1,y1 = p[a]
    x2,y2 = p[b]
    dist = ((x1-x2)**2+(y1-y2)**2)**0.5
    d[a][b] = min(dist, d[a][b])
    d[b][a] = min(dist, d[b][a])   
\end{lstlisting}

Алгоритм Флойда:

\begin{lstlisting}[language=Python]
for i in range(n):
    d[i][i] = 0    
for k in range(n):
    for i in range(n):
        for j in range(n):
            d[i][j] = min(d[i][j], d[i][k] + d[k][j])
\end{lstlisting}

Ищем $max$ в матрице игнорируя $Inf$:

\begin{lstlisting}[language=Python]
m = max(map(max, d))
print('-1' if m == Inf else m)
\end{lstlisting}

\centerline{\Huge Треугольники++ (Пифагор)}

Пифагор измерил длины сторон трех треугольников и записал их на клочке бумаги в случайном порядке. 
Через некоторое время он попытался разобрать свои записи и не смог 
определить какие размеры какому треугольнику соответствуют, но вспомнил, что, 
по крайней мере, два треугольника были подобными.

Даны размеры 9 сторон трех треугольников, 
записанные в случайном порядке. Указать - из каких сторон можно составить два подобных треугольника.

\begin{lstlisting}[language=Python]
l = [int(x) for x in input().split()]
n = len(l)
assert n == 9
\end{lstlisting}

Сортируем массив сторон по убыванию:

\begin{lstlisting}[language=Python]
l = sorted(l, reverse=True)
\end{lstlisting}

Пусть $i,j,k$ - индексы сторон первого треугольника.

$A,B,C$ - длины сторон первого треугольника.

$i_1,j_1,k_1$ - индексы сторон второго треугольника.

$a,b,c$ - длины сторон второго треугольника.

$\frac{A}{a} = \frac{B}{b} = \frac{C}{c}$

\begin{lstlisting}[language=Python]
for i in range(n-3):
    for j in range(i+1, n-2):
        for k in range(j+1, n):            
           A,B,C = l[i], l[j], l[k]
           for i1 in range(i+1, n):
                if i1 == j or i1 == k: continue                
                for j1 in range(i1+1, n-3):
                    if j1 == j or j1 == k: continue
                    a, b = l[i1], l[j1]
                    if A*b != B*a: continue                   
                    for k1 in range(j1+1, n-2):
                        if k1 == j or k1 == k: continue                       
                        c = l[k1] 
                        if A*c != C*a: continue 
                        print(A, B, C) 
                        print(a, b, c) 
                        sys.exit()
\end{lstlisting}


\end{document}