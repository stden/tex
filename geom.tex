\documentclass[a4paper,notitlepage,11pt]{article}

\usepackage{pscyr}
\usepackage[T2A]{fontenc}
\usepackage[utf8]{inputenc}
\usepackage[english,russian]{babel}
\usepackage[russian]{olymp}
\usepackage{tikz}
\usepackage{amssymb}
\usepackage{amsmath}
\usepackage{verbatim}
\usepackage{enumerate}
\usepackage{systeme}
\usepackage[makeroom]{cancel}

% IF YOU DON'T HAVE THIS FONT PACKAGE, JUST COMMENT THE NEXT THREE LINES!
\usepackage{pscyr}
\renewcommand{\rmdefault}{fha}
\renewcommand{\sfdefault}{fha}
% OR GET IT HERE: http://www.tex.uniyar.ac.ru/package/fonts/pscyr/

\renewcommand{\t}[1]{\ifmmode{\mathtt{#1}}\else{\texttt{#1}}\fi}
\newcommand{\s}[1]{`\t{#1}'}
\newcommand{\q}[1]{<<\t{#1}>>}
\newcommand{\eps}{\varepsilon}
\renewcommand{\t}{\texttt}
\renewcommand{\le}{\leqslant}
\renewcommand{\leq}{\leqslant}
\renewcommand{\ge}{\geqslant}
\renewcommand{\geq}{\geqslant}
\newcommand{\pow}{\^\relax}
\newcommand{\bs}{\mbox{$\backslash$}}
\newcommand{\seq}[2]{${#1}_{1}$, ${#1}_{2}$, $\ldots$, ${#1}_{#2}$}
\newcommand{\seqop}[3]{${#1}_{1} #3 {#1}_{2} #3 \ldots #3 {#1}_{#2}$}
\def\bbZ{\mathbb{Z}}
\DeclareMathOperator{\perm}{perm}

\contest
{Сессия по информатике, центр <<Интеллект>>}
{Разбор задач по геометрии}
{пятница, 10 июля 2015 года}

\binoppenalty=10000
\relpenalty=10000
%\parindent=0cm

\newcommand{\seriesshortname}{G}
\newcommand{\letterenumi}{{\alph{enumi}}}
\newcommand{\partlabelenumi}{{\Huge \texttt{\seriesshortname\letterenumi}}}
\renewcommand{\labelenumi}{{\Huge \texttt{\partlabelenumi.}}}
\newcommand{\infile}{\q{\seriesshortname\letterenumi.in} }
\newcommand{\outfile}{{\q{\seriesshortname\letterenumi.out}} }
\newcommand{\backref}[1]%
{\addtocounter{enumi}{-#1}\partlabelenumi\addtocounter{enumi}{+#1}}

\begin{document}

\centerline{\Huge Пересечение двух окружностей}

\begin{tikzpicture}[scale=3,cap=round]
  % Local definitions
  \def\costhirty{0.8660256}

  % Colors
  \colorlet{anglecolor}{green!50!black}
  \colorlet{color1}{red}
  \colorlet{color2}{blue}
  \colorlet{sincolor}{red}
  \colorlet{tancolor}{orange!80!black}
  \colorlet{coscolor}{blue}

  % Styles
  \tikzstyle{axes}=[]
  \tikzstyle{important line}=[very thick]
  \tikzstyle{information text}=[rounded corners,fill=red!10,inner sep=1ex]

  % The graphic
  \draw[style=help lines,step=0.5cm] (-1.4,-1.4) grid (1.4,1.4);
  
  \draw[style=important line,color1] (0,0) circle (1.2cm);
 
  \draw[style=important line,color2] (0.8, 0.4) circle (0.7cm);

  \begin{scope}[style=axes]
    \draw[->] (-1.5,0) -- (1.5,0) node[right] {$x$};
    \draw[->] (0,-1.5) -- (0,1.5) node[above] {$y$};

    \foreach \x/\xtext in {-1, -.5/-\frac{1}{2}, 1}
      \draw[xshift=\x cm] (0pt,1pt) -- (0pt,-1pt) node[below,fill=white]
            {$\xtext$};

    \foreach \y/\ytext in {-1, -.5/-\frac{1}{2}, .5/\frac{1}{2}, 1}
      \draw[yshift=\y cm] (1pt,0pt) -- (-1pt,0pt) node[left,fill=white]
            {$\ytext$};
  \end{scope}

  \draw (0.8,0) -- (intersection of 0,0--30:1cm and 0.8,0--0.8,1) coordinate (t);
    
  \draw (0,0) -- node [above left]
    {
      $\displaystyle \sqrt{x_2^2 + y_2^2}$
    } (t);

  \draw[xshift=1.85cm] node [right,text width=6cm,style=information text]
    {
      \textbf{Геометрия}

1. Геометрические объекты (понимаем, что они там есть: точки, прямые, отрезки, окружности)

2. Рисунок (упрощаем: перемещаем в начало координат, выкидываем лишнее)

3. Уравнения + неравенства (вид уравнения - вид объекта, параметы - конкретный объект)

4. Решаем уравнения + проверяем неравенства (+ точность вычислений)

5. Частные случаи => Интерпретации.

    };
\end{tikzpicture}

\systeme{
x^2 + y^2 = {R_1^2},
(x-x_2)^2 + (y-y_2)^2 = {R_2^2}
}

Раскрываем скобки во втором уравнении:
$$x^2 - 2x{x_2} + x_2^2 + y^2 - 2y{y_2} + y_2^2 = {R_2^2}$$

Вычитаем из второго уравнение первое и получаем уравнение прямой $ax + bx + c = 0$:
$$\cancel{x^2} - 2x{x_2} + x_2^2 + \cancel{y^2} - 2y{y_2} + y_2^2 - \cancel{x^2} - \cancel{y^2} = {R_2^2} - {R_1^2}$$

$$\underbrace{-2{x_2}}_{= a}x + \underbrace{- 2{y_2}}_{= b}y + \underbrace{x_2^2 + y_2^2 + {R_1^2} - {R_2^2}}_{= c} = 0$$

Из $ax + by + c = 0$ выражаем $y = \frac{-c-ax}{b}$: 
$x^2 + (\frac{-c-ax}{b})^2 y^2 = R_1^2$

Раскрываем скобки и домножаем на $b^2$

$$\underline{\underline{{b^2}{x^2}}} + c^2 + \underline{2cax} + \underline{\underline{{a^2}{x^2}}} = {R_1^2}{b^2}$$

Квадратное уравнение относительно $x$:

$$\underbrace{(a^2+b^2)}_{A}x^2 + \underbrace{2ca}_{B}x + \underbrace{(c^2 - {R_1^2}{b^2})}_{C} = 0$$

Решаем квадратное уравнение:

Дискриминант: $D = B^2 - 4AC$


$$x_{1,2} = \frac{-B \pm \sqrt{D}}{2A}$$

\end{document}
